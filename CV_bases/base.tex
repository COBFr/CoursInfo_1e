\documentclass[frenchb]{scrartcl}
\usepackage[utf8]{inputenc}\usepackage[T1]{fontenc}
\usepackage[textwidth=17.5cm,textheight=25cm,left=2cm,top=2cm]{geometry}
\usepackage[urlcolor=blue,colorlinks=true]{hyperref}
\usepackage{xspace,paralist,multicol,scrpage2,lastpage,lmodern,moreverb,babel}

\setlength\columnseprule{0.4pt}\setlength\multicolsep{1pt}
\pagestyle{scrheadings}\clearscrheadfoot
\ifoot[]{\textsc{{\small COB/RC}}}\cfoot[{\small \pagemark}]{{\small }}\ofoot[]{{\small p.~\thepage\,/\,\pageref{LastPage}}}
\ihead[]{{\small Informatique}}\chead[]{{\small Classe 1D}}\ohead[]{{\small \number\day/\,\number\month/\,\number\year}}

\begin{document}
\begin{center}
	{\huge Faire un CV}\par
	\textbf{\textit{Texte de base avant la mise en forme}}
\end{center}\smallskip\noindent\hrulefill{}\\[1ex]

\begin{multicols}{2}

\subsubsection*{Objectif}
Avoir les données de base du CV avant de les mettre en forme.

%\subsubsection*{Délais}
%\begin{center}
%	\large\bfseries
%	Mardi 13 janvier 2015
%\end{center}

\subsubsection*{Pas à pas}
Préparation du travail:
\begin{compactenum}
	\item Ouvrir educanet.ch;
	\item aller sur \texttt{Espace privé}, \texttt{Classeur}, dossier \texttt{CV};
	\item télécharger le document que nous avons commencé en cours;
	\item le renommer\\ \texttt{1D\_Prénom\_CV\_base\_devoir.odt};
	\item afficher les caractères non imprimables;
\end{compactenum}

\subsubsection*{Contenu}
Ce que vous devez mettre dans votre document:

N'oubliez pas de remplacer les "<TAB>" par le caractère tabulation.
\end{multicols}

\begin{center}\begin{boxedverbatim}
	Prénom NOM
	Rue 123
	4567 Ville
	026 123 45 67
	079 876 54 32
	14nnnnppp@belluard.educanet2.ch
	Australien, permis C
	Né le 30 février 1919
	Parcours scolaire
	2008 - 2014<TAB>École primaire, le Botzet, Fribourg
	Stages
	2013<TAB>Carrossier, AB Auto, Fribourg
	2013<TAB>Peintre en voiture, Carrosserie BC, Avry
	Langues
	Français<TAB>Langue maternelle
	Italien<TAB>Parlé à la maison
	Allemand<TAB>Connaissances scolaires de base
	Connaissances informatiques
	Internet<TAB>Bonnes connaissances
	Traitement de texte<TAB>Bonnes connaissances
	Intérêts personnels
	Loisirs<TAB>Musique, marche en montagne
	Sports<TAB>Ski, aïkido
\end{boxedverbatim}
\end{center}

\begin{multicols}{2}
\subsubsection*{Remarques}
Ce n'est pas un exercice!
Les données que vous mettez dans ce document doivent être vos données personnelles, correctes.

Vous devez faire attention aux points suivants:
\begin{compactenum}
	\item toutes les données demandées doivent être renseignées et vérifiées
		(à moins qu'elles n'existent pas, ex. pas de téléphone à la maison);
	\item vérifiez bien l'orthographe;
	\item mettre au féminin ce qui doit l'être;
	\item l'usage des majuscules doit être respecté (première lettre des noms propres, nom de famille entièrement en majuscule, accentuée au besoin\dots);
	\item les espaces doivent être corrects (pas de doubles espaces, pas de doubles tabulations, pas d'espaces ou de tabulation en fin de ligne\dots).
	\item mettre le document dans votre classeur educanet, dans le dossier CV.
\end{compactenum}

\end{multicols}
\end{document}
